\documentclass[a4paper,12pt]{article}
\usepackage{float}
\newcommand{\myparagraph}[1]{\paragraph{#1}\mbox{}\\}
\usepackage{lipsum}

\usepackage{array}
\usepackage[export]{adjustbox}
\usepackage[table,xcdraw]{xcolor}
\ProvidesPackage{easygreekmanolis}[2019/02/12 Easy Greek by Manolis]
\usepackage{graphicx,wrapfig}
\graphicspath{ {./images/} }

\RequirePackage{fontspec}
\RequirePackage{polyglossia}
\setdefaultlanguage{greek}
\setotherlanguage{english}
\setmainfont{Times New Roman}

\newcommand{\gr}[1]{\begin{greek}#1\end{greek}}
\newcommand{\en}[1]{\begin{english}#1\end{english}}

\setlength\parskip{1ex}

\begin{document}
	
	\title{\Large\textbf{{ΑΝΑΠΤΥΞΗ ΥΒΡΙΔΙΚΗΣ ΕΦΑΡΜΟΓΗΣ ΕΞΕΙΔΙΚΕΥΜΕΝΩΝ ΑΘΛΗΤΙΚΩΝ ΠΡΟΓΡΑΜΜΑΤΩΝ}}}
	
	\author{\large{{ΓΕΩΡΓΙΟΣ ΔΟΙΤΣΙΝΗΣ}}}
	
	\date{ΙΟΝΙΟ ΠΑΝΕΠΙΣΤΗΜΙΟ \\ ΤΜΗΜΑ ΠΛΗΡΟΦΟΡΙΚΗΣ \\[5mm] \today}
	
	\maketitle
	
	\newpage
	 \section{Ευχαριστίες}

		Για αρχή θα ήθελα να ευχαριστήσω θερμά τον επιβλέποντα καθηγητή μου κ. Μιχαήλ 
		Στεφανιδάκη ο οποίος με καθοδήγησε ορθά και ήταν πάντα δίπλα μου προκειμένου να
		καταφέρω να φέρω εις πέρας την παρούσα εργασία. 
		Ένα μεγάλο ευχαριστώ αξίζει στην οικογένια μου που με στηρίζει κυριολεκτικά από τα
		πρώτα βήματα μου έως και σήμερα με υπομονή, αγάπη, φροντίδα και συμπαράσταση.
		
 	\newpage		
 	\section{Σύνοψη}

		Με τη μεγάλη εξάπλωση των κινητών τηλεφώνων το μεγλύτερο μέρος του πληθυσμού έρχεται πλέον σε επαφή  
		με ένα πλήθος νέων εφαρμογών που λίγα χρόνια πριν ηταν διαθέσιμες μόνο μέσω προσωπικού υπολογιστή.

		Σε αυτήν την εργασία θα παρουσιαστεί η ανάπτυξη μιας εφαρμογής που θα διδάσκει το άθλημα της πυγμαχίας. Η εφαρμογή μπορεί και τρέχει
		σε κινητές συσκευές με λειτουργικό συστημα Android , iOS και Windows.
	 		
 	\newpage
 	\section{Abstract} 
		
		With the large spread of mobile phones, most of the population is now in contact
		with a large number of new applications that a few years ago were only available on a personal computer.
		
		This work will present the development of an application that will teach the sport of boxing. The application runs
		on Android, iOS, and Windows operating systems.

	\newpage	
	\tableofcontents
	\listoffigures
	\listoftables

	\newpage
	\section{Εισαγωγή}

		\subsection{Περίληψη}

			Η πυγμαχία, κοινώς μποξ είναι ένα από τα πιο δημοφιλή αγωνίσματα και μαχητικές πολεμικές τέχνες, που στηρίζεται στην ικανότητα των αντιπάλων 
			να αντικρούσουν μόνο με τις γροθιές τους ο ένας τον άλλο και μέσω εύστοχων χτυπημάτων να κερδίσουν τη μεταξύ τους αναμέτρηση.

			Η παρούσα εργασία έχει ως στόχο τη δημιουργία μιάς υβριδικής εφαρμογής για Android 
			και iOS η οποία απευθύνεται στα άτομα που θέλουν να γυμναστούν μαθαίνοντας τις κινήσεις του αθλήματος της Πυγμαχίας και να ακολουθήσουν 
			μια καθημερινή ρουτίνα γυμναστικής και εκπαίδευσης.

		\subsection{Κίνητρο για τη διεξαγωγή της εργασίας}	

			Υπάρχουν πολλές εφαρμογές που αναφέρονται στη γυμναστική και στη σωστή διατροφή, άλλωστε απο εκεί προέκυψε και η ιδέα της εργασίας, 
			παρόλα αυτά δεν υπάρχει κάποια παρόμοια, που να διδάσκει ένα άθλημα η μία τέχνη και παράλληλα να προάγει έναν υγιεινό τρόπο ζωής. 


		\subsection{Σκοπός και στόχοι εργασίας}	
			
			Σκοπός της εργασίας είναι αρχικά να απενοχοποιήσει το παρεξηγημένο 
			άθλη-μα της πυγμαχίας από την κοινή αίσθηση ότι είναι ένα βάρβαρο άθλημα 
			που συνεπάγεται με σωματικούς τραυματισμούς αλλά και πνευματικά προβλήματα.
			Να δείξει στους χρήστες οτι παρέχει σωματικά και πνευματικά οφέλη και 
			να τους δώσει τη δυνατότητα να εξασκηθούν, να καλυτερέψουν τη φυσική τους κατάσταση 
			και να χάσουν βάρος. Τέλος να τους παρέχει οπτικά διαγράμματα ώστε να βλέπουν την πρόοδο τους και έτσι να τους δίνει κίνητρο για να συνεχίσουν
			να τη χρησιμοποιούν.   


	\newpage
	\section{Υβριδικές mobile εφαρμογές}
	
		\subsection{Τι είναι μία φορητή εφαρμογή}
		
			Μία φορητή εφαρμογή (ή αλλιώς mobile app) είναι η εφαρμογή λογισμικού σχεδιασμένη να
			τρέχει σε smartphone, υπολογιστές tablet και άλλες φορητές συσκευές. Είναι διαθέσιμη στο κοινό μέσω πλατφορμών διανομής εφαρμογών, οι οποίες συνήθως λειτουργούν από τον
			ιδιοκτήτη του φορητού λειτουργικό συστήματος, όπως το Apple App Store, Google Play, BlackBerry App World.
			Τα Mobile apps, αρχικά είχαν στόχο την προσφορά στη γενική παραγωγικότητα του κοινού και
			την ανάκτηση πληροφοριών, συμπεριλαμβανομένων εφαρμογών για e-mail, ημερολόγιο, κατάλογο
			επαφών, χρηματιστηριακές αγορές και πληροφορίες για τον καιρό. Ωστόσο, η δημόσια ζήτηση και
			η διαθεσιμότητα των εργαλείων ανάπτυξης οδήγησε με γρήγορους ρυθμούς σε επέκταση και άλλων
			κατηγοριών, όπως παιχνίδια, αυτοματισμούς εργοστασίων, GPS και location-based υπηρεσίες,
			banking, εξέλιξη παραγγελιών, καθώς και στις αγορές εισιτηρίων.
		
		\subsection{Τι σημαίνει υβριδική εφαρμογή}
		
			Υβριδικό, εξ ορισμού, είναι οτιδήποτε προέρχεται από ετερογενείς πηγές ή αποτελείται από στοιχεία διαφορετικών ή ασυμβίβαστων όρων. Υβριδική εφαρμογή είναι αυτή που 
			είναι γραμμένη με την ίδια τεχνολογία, που χρησιμοποιείται για ιστότοπους και εφαρμογές ιστού για κινητά και που φιλοξενείται ή τρέχει μέσα σε μία κινητή συσκευή. 
				
			Η έννοια είναι πολύ απλή. Σχεδόν κάθε κινητό λειτουργικό σύστημα διαθέτει ένα API για την ανάπτυξη εφαρμογών. 
			Αυτό το API αποτελείται από ένα στοιχείο που ονομάζεται Web View. Web view είναι συνήθως ένα πρόγραμμα περιήγησης που λειτουργεί εντός του πεδίου μιας εφαρμογής για κινητά. 
			Αυτό το πρόγραμμα περιήγησης εκτελεί τους HTML, CSS και Javascript κώδικες. Αυτό σημαίνει ότι μπορεί κάποιος να δημιουργήσει μια ιστοσελίδα χρησιμοποιώντας τις προηγούμενες 
			τεχνολογίες και στη συνέχεια να την εκτελέσει μέσα στην εφαρμογή του.
							
			Ο χρήστης μπορεί να χρησιμοποιήσετε τις ίδιες γνώσεις ανάπτυξης ιστοσελίδων για να δημιουργήσει native-hybrid κινητές εφαρμογές (εδώ, ο όρος native 
			αναφέρεται στην εγκατάσταση ενός αρχείου συγκεκριμένη μορφής για τη συγκεκριμένη πλατφόρμα στη συσκευή αφού έχει συσκευαστεί μαζί με άλλα εργαλεία), για παράδειγμα:
			\begin{itemize}

				\item Η Android χρησιμοποιεί Android Application Package (.apk)				
				\item Η iOS χρησιμοποιεί το iPhone Application Archive (.ipa)
				\item Η Windows Phone χρησιμοποιεί Application Package (.xap)
				
			\end{itemize}
			Το πακέτο / πρόγραμμα εγκατάστασης αποτελείται από ένα κομμάτι native κώδικα που προετοιμάζει την ιστοσελίδα
			και κάποια στοιχεία που απαιτούνται για την εμφάνιση του περιεχομένου της ιστοσελίδας.

			Η έννοια της εμφάνισης μιας ιστοσελίδας μέσα στο κιβώτιο εφαρμογών για κινητά ονομάζεται Υβριδική εφαρμογή.

			Τα πλεονεκτήματα της ανάπτυξης υβριδικών εφαρμογών είναι:
		
			\begin{itemize}
				\item Πρόκειται ουσιαστικά για εφαρμογές ιστού που είναι αποθηκευμένες σε επίσημα καταστήματα εφαρμογών.
			
				\item Μπορεί να χρησιμοποιηθεί μαζί με οποιοδήποτε JavaScript framework / βιβλιοθήκη .
				
				\item Ο κώδικας είναι σε μεγάλο βαθμό κοινόχρηστος (shaerable) σε όλες τις πλατφόρμες.
				
				\item Πρόσβαση σε μητρικές (native) λειτουργίες (για παράδειγμα, κάμερα, επιταχυνσιόμετρο, λίστα επαφών).
			\end{itemize}
			
			Αλλά όπως πλέονεκτηματα υπάρχουν και μειωνεκτήματα:
			
			\begin{itemize}
				\item Αντιμετωπίζουν προβλήματα επιδόσεων και κατανάλωσης μνήμης, καθώς οι προβολές ιστού (web views) ευθύνονται για οτιδήποτε εμφανίζεται στην οθόνη.
				
				\item Πρέπει να μιμούνται όλα τα μητρικά στοιχεία UI πάνω από μία ενιαία προβολή ιστού.
				
				\item Πιο δύσκολο να γίνου δεκτές και να δημοσιευτούν στο App Store.
				
				\item Συνήθως χρειάζονται περισσότερο χρόνο για να έχουν διαθέσιμες μητρικές λειτουργίες.
				
				\item Ορισμένες λειτουργίες ή σχέδια δεν υποστηρίζονται και στις δύο συσκευές( Android και iOS), πράγμα που απαιτεί τροποποίηση.
			
			\end{itemize}
		\subsection{Λίγα λόγια για το Apache Cordova}
		
			Το Apache Cordova είναι ένα λογισμικό που ενώνει μια εφαρμογή Web
			με μια native εφαρμογή. Ο δικτυακός τόπος Apache Cordova αναφέρει ότι:
			"Το Apache Cordova είναι μια πλατφόρμα για τη δημιουργία native εφαρμογών κινητής τηλεφωνίας
			HTML, CSS και JavaScript. "


			\begin{figure}[!htb]
				\caption{Αρχιτεκτονική Apache Cordova.}
				\vspace*{0.5cm}
				\centering
				\includegraphics[width=0.63\linewidth]{Cordova} 
			  \end{figure}
			  
			\newpage
			Η προηγούμενη εικόνα απεικονίζει την αρχιτεκτονική υψηλού επιπέδου του Apache Cordova. 
			Το Apache Cordova δεν ενώνει μόνο την εφαρμογή Web με τη native εφαρμογή, αλλά επίσης
			παρέχει ένα σύνολο API γραμμένο σε JavaScript για να αλληλεπιδράσει με τις native λειτουργίες της
			συσκευής. Μπορεί κανείς να χρησιμοποιήσει τη JavaScript για να αποκτήσει πρόσβαση στην κάμερά, να τραβήξει μια φωτογραφία ή να μιλήσει με το Bluethooth της
			συσκευής και να πάρει τη λίστα με της συσκεύες στη γύρω περιοχή. Στην τελευταία περίπτωση, η Cordova διαθέτει κάποια API που αλληλεπιδρούν 
			με το Web View χρησιμοποιώντας τη JavaScript και στη συνέχεια μιλάει στη συσκευή στη μητρική της γλώσσα παρέχοντας έτσι μια γέφυρα μεταξύ τους.
			
			Η παρακάτω εικόνα απεικονίζει την αρχιτεκτονική Plugin υψηλού επιπέδου:

			\begin{center}
				
			\end{center}
			\begin{figure}[!htb]
				\caption{Αρχιτεκτονική Cordova Plugin.}
				\vspace*{0.5cm}
				\centering
			\includegraphics[width=0.9\linewidth]{Cordova2} 			  
			\end{figure}
			
			\newpage
			Τα Plugin (πρόσθετα) πάντα αποτελούνται από δύο μέρη. Το πρώτο είναι ένα τμήμα JavaScript που εκτελείται μέσα στο WebView το οποίο εκθέτει ένα API στην υβριδική εφαρμογή. 
			Το δέυτερο μέρος είναι συγκεκριμένο για κάθε πλατφόρμα και είναι γραμμένο στη μητρική της γλώσσα, π.χ. Java 
			για Android και Objective-C για iOS. Τα native API ελέγχονται απο το δέυτερο μέρος.

			Τα Plugin αποτελούν αναπόσπαστο μέρος του οικοσυστήματος της Cordova. Παρέχουν μια διασύνδεση για την Cordova και τα native 
			συστατικά για να επικοινωνούν μεταξύ τους και να συνδέονται με τα τυπικά API της συσκευής. Αυτό επιτρέπει στο χρήστη να εφαρμόζει το 
			native κώδικα από τη JavaScript. Το πρόγραμμα Apache Cordova διατηρεί ένα σύνολο από plugins που ονομάζονται Core Plugins (κεντρικά πρόσθετα). 
			Αυτά τα core plugins παρέχουν στην εφαρμογή πρόσβαση σε δυνατότητες συσκευών όπως μπαταρία, κάμερα, επαφές κ.λπ.

			Εκτός από τα core plugins, υπάρχουν πολλά plugins τρίτων που προσφέρουν πρόσθετες συνδέσεις σε λειτουργίες που δεν είναι 
			απαραίτητα διαθέσιμες σε όλες τις πλατφόρμες. Επίσης κάθε χρήστης μπορεί να δημιουργήσει δικά του Plugins για συγκεκριμένες λειτουργίες που θέλει να έχει
			η εφαρμογή του, τα οποία στη συνέχεια μπορεί να τα δημοσιέυει στην ιστοσελίδα της Apache Cordova. Πράττοντας έτσι άλλοι χρήστες μπορόυν να τα εισάγουν
			στην εφαρμογή τους, να εντοπίσουν σφάλματα και να τα επεκτείνουν, διευκολύνοντας έτσι τους επόμενους.
		\newpage
		\subsection{Λίγα λόγια για τo Ionic Framework}
			Το Ionic είναι ένα πλήρες SDK ανοιχτού κώδικα για ανάπτυξη υβριδικών εφαρμογών για κινητά που δημιουργήθηκε από τους Max Lynch, Ben Sperry και Adam Bradley από την Drifty Co. το 
			2013. Η αρχική έκδοση κυκλοφόρησε το 2013 και χτίστηκε πάνω από τα AngularJS και Apache Cordova. Ωστόσο, η τελευταία έκδοση δημιουργήθηκε εκ νέου ως ένα σύνολο στοιχείων Web, 
			επιτρέποντας στο χρήστη να επιλέξει οποιοδήποτε Framework, όπως το Angular, React ή Vue.js. Μέσω της Cordova μπορεί κανείς να έχει πρόσβαση στα API της συσκευής χρησιμοποιώντας 
			μια βιβλιοθήκη όπως η ngCordova και να τα συνδιάζει με στοιχεία διεπαφής χρήστη του Ionic.

			Το Ionic Framework, στον πυρήνα του, αποτελείται από τέσσερα μέρη:

			\begin{itemize}
				\item Ένα \textbf{stylesheet} που ορίζει μια βελτιστοποιημένη διάταξη για κινητά. Αυτή η διάταξη χρησιμεύει ως βάση για την εφαρμογή.
				
				\item Το \textbf{AngularJS} module (δομοστοιχείο) oρίζει τις οδηγίες, τα πρότυπα πλοήγησης και τις βέλτιστες πρακτικές, ώστε να μην χρειάζεται να το κάνει ο χρήστης.
				
				\item Ενα πρόγραμμα γραμμής εντολών (\textbf{CLI}-Command Line Interface) που δέχεται την εισαγωγή κειμένου για την εκτέλεση λειτουργιών του λειτουργικού συστήματος.
				Στη συγκεκριμένη περίπτωση λειτουργεί σαν ένα είδος μεσολάβησης για το Cordova και το Gulp CLI.
				
				\item Το τελευταίο στοιχείο είναι ένα πρόσθετο πληκτρολογίου \textbf{keyboard plugin}. Αν και τεχνικά δεν απαιτείται, το πρόσθετο παρέχει περισσότερες πληροφορίες σχετικά με την τρέχουσα κατάσταση της εφαρμογής.
				
			\end{itemize}

			Παρακάτω υπάρχει μια εικόνα με την αρχιτεκτονική μιας Ionic εφαρμογής:
			\begin{figure}
				\caption{Αρχιτεκτονική Ionic.}
				\vspace*{0.5cm}
				\centering
				\includegraphics[width=0.8\linewidth]{ionic} 
			\end{figure}
		
		\newpage
		\subsection{Τι σημαίνει ο όρος SDK}
			
			Το SDK (Software Development Kit) σημαίνει "Κιτ ανάπτυξης λογισμικού". Σκεφτείτε το σαν να κατασκεβάζετε ένα μοντέλο αυτοκινήτου. Κατά την κατασκευή αυτού του μοντέλου,
			απαιτείται ένα πλήρες σύνολο στοιχείων, συμπεριλαμβανομένων των τεμαχίων του κιτ, των εργαλείων που απαιτούνται για την τοποθέτησή τους, των οδηγιών συναρμολόγησης και ούτω
			καθεξής.
			
			Ένα SDK ή devkit λειτουργεί με τον ίδιο τρόπο, παρέχοντας ένα σύνολο εργαλείων, βιβλιοθηκών, σχετικής τεκμηρίωσης, δειγμάτων κώδικα, διεργασιών ή οδηγών που επιτρέπουν στους
			προγραμματιστές να δημιουργούν εφαρμογές λογισμικού σε μία συγκεκριμένη πλατφόρμα. Εάν ένα API είναι ένα σύνολο δομικών στοιχείων που επιτρέπουν τη δημιουργία ενός στοιχείου,
			ένα SDK είναι ένα πλήρες εργαστήριο, διευκολύνοντας τη δημιουργία στοιχείων που είναι αδύνατο να δημιθουργηθούν με απλά API.
			
			Τα SDK είναι οι πηγές προέλευσης για σχεδόν κάθε πρόγραμμα με τον οποίο μπορεί να αλληλεπιδράσει ένας σύγχρονος χρήστης με άλλες εφαρμογές.
		\newpage
		\subsection{Τι σημαίνει ο όρος API}
		
			Ένα API είναι απλά μία διασύνδεση που επιτρέπει στο λογισμικό να αλληλεπιδρά με άλλα λογισμικά. Αυτό είναι μέρος του ονόματός του - API, Application Programming Interface - και 
			αποτελεί βασικό στοιχείο της λειτουργικότητάς του.
			
			Τα API διατίθενται σε πολλά σχήματα και μεγέθη. Το πρόγραμμα περιήγησης που ένας χρήστης θα χρησιμοποιήσει για οποιονδήποτε ιστότοπο επισκευτεί κατά πάσα πιθανότητα 
			χρησιμοποιεί μία ποικιλία συνόλων API για να μετατρέπει τις εντολές χρηστών σε χρήσιμες λειτουργίες, να ζητά δεδομένα από διακομιστές (servers), να απεικονίζει αυτά τα δεδομένα σε 
			κατάλληλη μορφή για να μπορεί να δει ο χρήστης και να επικυρώνει την απόδοση των αιτημάτων τους.
			
			Ακόμα και κάτι τόσο απλό όσο η αντιγραφή και η επικόλληση σε έναν υπολογιστή χρησιμοποιεί ένα API. Η αντιγραφή κειμένου μετατρέπει το πάτημα ενός κουμιού σε μία εντολή, τα δεδομένα 
			αποθηκεύονται στη μνήμη RAM στο πρόχειρο χρησιμοποιώντας ένα API, τα δεδομένα μεταφέρονται στη συνέχεια από μία εφαρμογή σε άλλη χρησιμοποιώντας το ίδιο API και τελικά τα δεδομένα 
			αποδίδονται κατά την επικόλληση χρησιμοποιώντας άλλο API. 
			
			Στον παγκόσμιο ιστό, το API αναλαμβάνει μία ελαφρώς διαφορετική λειτουργία. Τα Web API επιτρέπουν την αλληλεπίδραση μεταξύ διαφορετικών συστημάτων, συχνά για συγκεκριμένες 
			περιπτώσεις χρήσης. Για παράδειγμα, όταν οι χρήστες αλληλεπιδρούν με το Facebook, χρησιμοποιούν ένα API για να σχολιάσουν, να αποθηκεύσουν τα δεδομένα τους, να ακολουθήσουν έναν 
			χρήστη, να διαγράψουν τα post κ.ο.κ. Τελικά, ένα web API είναι απλά ένα σύνολο οδηγιών, όπως ακριβώς το API του προσωπικού υπολογιστή, αλλά βασίζεται στο χώρο του web.
			
			Ίσως το πιο σημαντικό είναι το γεγονός ότι τα API επιτρέπουν τη συνέπεια. Στα πρώτα χρόνια του προγραμματισμού, οι εντολές και οδηγίες ήταν χαλαρά κωδικοποιημένες και σπάνια τεκμηριωμένες.   
			Με την εμφάνιση των σύγχρονων υπολογιστών, τα API επέτρεψαν τη συνεπή κωδικοποίηση σε σταθερά περιβάλλοντα, επιτρέποντας αντιγράψιμες λειτουργίες να παραδίδονται απαράλλακτες κάθε φορά που 
			υποβάλεται μία αίτηση με αξιοπιστία και προβλεψιμότητα.
		
		
			
		\newpage
		\subsection{Λίγα λόγια για την Angular}

			Η Angular είναι το πιο δημοφιλές πλαίσιο εφαρμογών ιστού ανοιχτού κώδικα (open-source web application framework) της Google και είναι βασισμένο στην TypeScript. Η πρώτη της έκδοση αναπτύχθηκε
			το 2010 και ονομάστηκε AngularJs. Η δέυτερη έκδοση αναπτύχθηκε το 2014. Σήμερα όλες οι εκδόσεις μετά τη δέυτερη μέχρι την τελευταία που είναι η έβδομη, ονομάζονται Angular. 
			Κυριώς χρησιμοποιείται για τη δημιουργία Εφαρμογών Ενιαίας Σελίδας (SPA-Single Page Application). Με τον όρο SPA αναφερόμαστε σε μία εφαρμογή ιστού ή μία ιστοσελίδα πού αλληλεπιδρά με το χρήστη μέσω της δυναμικής επανεγγραφής 
			της τρέχουσας σελίδας αντί της φόρτωσης ολόκληρων νέων σελίδων απο το διακομιστή. Παράδειγμα τέτοιας σελίδας είναι η GMAIL, όπου χωρίς ανανέωση, ανοίγει όλες τις λεπτομέρειες της σελίδας. 
			Έχει μόνο μία σελίδα HTML, η οποία ζητά περιεχόμενο άλλων σελίδων από το διακομιστή. Αποτέλεσμα είναι πολύ πιο μικροί χρόνοι φόρτωσης. Οι εφαρμογές Angular κατασκευάζονται από εξαρτήματα τα οποία
			μπορούν να ενσωματώνονται μεταξύ τους και έτσι η δημιουργία εφαρμογών γίνεται πολύ εύκολη.

		\newpage
		\section{Φάσεις Ανάπτυξης Εφαρµογής}
		
		\subsection{Ανάλυση}

		Στην ενότητα αυτή παρουσιάζεται η φάση της ανάλυσης του συστήµατος καθώς και
		µια αρχική προσέγγιση της εφαρµογής που πρόκειται να αναπτυχθεί. Αποτελεί την
		πρώτη φάση και µια από τις σηµαντικότερες ώστε να προχωρήσει κάποιος ορθά στη
		σχεδίαση και στην υλοποιηση. Στην παρούσα φάση της εφαρµογής καθορίστηκαν οι κύριοι στόχοι της
		καθώς και συγκεντρωθήκαν οι απαιτήσεις χρηστών.
		
		Παρακάτω ακολουθεί ένας
		πίνακας ερωταπαντήσεων που συνετέλεσε στην περαιτέρω οργάνωση
		της δοµής του έργου:

			
		\begin{table}[htb]
			\centering
			\caption{Ανάλυση βασικών ερωτημάτων}
			\vspace*{0.2cm}			
			\label{tab:my-table}
			\resizebox{\textwidth}{!}{%
			{\renewcommand{\arraystretch}{1.5}

			\begin{tabular}{cc}
			\textbf{Ερώτημα}                                                                                                  & \textbf{Απάντηση}                                                                                             \\ \hline
			\multicolumn{1}{|c|}{Ποιά είναι η ηλικία χρηστών;}                                                                & \multicolumn{1}{c|}{Ολες οι ηλικές επιτρέπονται}                                                                               \\ \hline
			\multicolumn{1}{|c|}{Τι γλώσσα µιλούν;}                                                                           & \multicolumn{1}{c|}{Αγγλικά}                                                                                  \\ \hline
			\multicolumn{1}{|c|}{\begin{tabular}[c]{@{}c@{}}Έχουν χρησιµοποιήσει παλιότερα\\ smartphone;\end{tabular}}        & \multicolumn{1}{c|}{Ναι}                                                                                      \\ \hline
			\multicolumn{1}{|c|}{\begin{tabular}[c]{@{}c@{}}Έχουν χρησιµοποιήσει παρόµοιες\\ εφαρµογές;\end{tabular}}         & \multicolumn{1}{c|}{Πιθανώς}                                                                                      \\ \hline
			\multicolumn{1}{|c|}{Υφίσταται φυλετικός διαχωρισµός;}                                                            & \multicolumn{1}{c|}{Όχι}                                                                                      \\ \hline
			\multicolumn{1}{|c|}{Χρειάζεται πρόσβαση στο ∆ιαδίκτυο;}                                                          & \multicolumn{1}{c|}{Ναι, για την εγγραφή}                                                                      \\ \hline
			\multicolumn{1}{|c|}{\begin{tabular}[c]{@{}c@{}}Απαιτείται πρόσβαση σε τοποθεσία\\ χρήστη;\end{tabular}}          & \multicolumn{1}{c|}{Οχι}                                                                                      \\ \hline
			\multicolumn{1}{|c|}{\begin{tabular}[c]{@{}c@{}}Απαιτήται πρόσβαση στις φωτογραφίες ή τις επαφες του \\χρήστη;\end{tabular}} & \multicolumn{1}{c|}{Οχι}                                                                               \\ \hline
			\multicolumn{1}{|c|}{\begin{tabular}[c]{@{}c@{}}Απαιτήται η οπτική αναπαράσταση της \\προόδου του χρήστη;\end{tabular}} & \multicolumn{1}{c|}{Ναι}                                                                               \\ \hline
			\multicolumn{1}{|c|}{\begin{tabular}[c]{@{}c@{}}Πώς θα γίνει η διανοµή του τελικού\\ προιόντος;\end{tabular}}     & \multicolumn{1}{c|}{\begin{tabular}[c]{@{}c@{}}Προσωπική χρήση στα πλαίσια\\ πτυχιακής εργασίας\end{tabular}} \\ \hline
			\multicolumn{1}{|c|}{\begin{tabular}[c]{@{}c@{}}Πόσο χρόνο διαθέτουµε για την\\ ανάπτυξη του έργου;\end{tabular}} & \multicolumn{1}{c|}{Εξι µήνες}                                                                               \\ \hline

			\end{tabular}%
			}
			}
		\end{table}
		\newpage
		Έπειτα καθορίστηκαν τα τµήµατα υλοποίησης του έργου και αποφασίστηκε ο
		χρονοπρογραµµατισµός του. Παράλληλα, τα παραπάνω τµήµατα τοποθετήθηκαν
		σε χρονική σειρά και καθορίστηκε η χρονική τους διάρκεια. 
		
		Στον παρακάτω πίνακα
		ακολουθεί το χρονοδιάγραµµα εκπόνησης έργου:
		
		\begin{table}[htb]
			\centering
			\caption{Πλάνο και Χρονοπρογραµµατισµός Έργου}
			\vspace*{0.5cm}
			\label{tab:my-table}
			\resizebox{\textwidth}{!}{%
			{\renewcommand{\arraystretch}{2.5}
			\begin{tabular}{|l|l|l|l|l|l|l|}
			\hline
			\textbf{Πλάνο Εργασιών}                             & \textbf{Απρίλιος}        & \textbf{Μάιος}           & \textbf{Ιούνιος}         & \textbf{Ιούλιος}         & \textbf{Αύγουστος}       & \textbf{Σεπτέµβριος}     \\ \hline
			\textbf{Φάση Α:Ανάλυση}                             & \cellcolor[HTML]{4E8EF7} &                          &                          &                          &                          &                          \\ \hline
			Καταγραφή βασικών ιδεών                             & \cellcolor[HTML]{D14A47} &                          &                          &                          &                          &                          \\ \hline
			Χρονο προγραµµατισµός και πλάνο έργου               & \cellcolor[HTML]{D14A47} &                          &                          &                          &                          &                          \\ \hline
			\textbf{Φάση Β:Σχεδίαση}                            &                          & \cellcolor[HTML]{4E8EF7} &                          &                          &                          &                          \\ \hline
			Κατηγοριοποίηση και επιλογή περιεχομένου            &                          & \cellcolor[HTML]{D14A47} &                          &                          &                          &                          \\ \hline
			\textbf{Φαση Γ:Υλοποίηση}                           &                          &                          & \cellcolor[HTML]{4E8EF7} & \cellcolor[HTML]{4E8EF7} & \cellcolor[HTML]{4E8EF7} &                          \\ \hline
			Συγκεντρωση και επεξεργασία πρωτογενούς υλικού      &                          &                          & \cellcolor[HTML]{D14A47} & \cellcolor[HTML]{D14A47} & \cellcolor[HTML]{D14A47} &                          \\ \hline
			\textbf{Φαση Δ:Ελεγχος}                             &                          &                          &                          &                          &                          & \cellcolor[HTML]{4E8EF7} \\ \hline
			Ελεγχος λειτουργίας εφαρμογής και απατήσεων χρηστών &                          &                          &                          &                          &                          & \cellcolor[HTML]{D14A47} \\ \hline
			Διωρθώσεις για την παραγωγή του τελικού προϊόντος   &                          &                          &                          &                          &                          & \cellcolor[HTML]{D14A47} \\ \hline
			\end{tabular}%
			}
			}
		\end{table}
		\newpage
		\subsection{Σχεδίαση}
		Η φάση της σχεδίασης αποτελεί τη δεύτερη φάση ανάπτυξης της εφαρµογής. Οι
		γενικοί στόχοι και οι αρχές που καθορίστηκαν στη φάση της ανάλυσης µετατρέπονται
		σε µια ολοκληρωµένη αναλυτική περιγραφή της εφαρµογής. Σε αυτή τη φάση
		καθορίζεται και ή αρχιτεκτονική του συστήµατος.

		\textbf{Απαιτήσεις σχεδιασµού:}

		Κατά το σχεδιασµό µιας εφαρµογής υπάρχουν κάποιες απαιτήσεις που πρέπει να
		έχει. Τα εργαλεία που χρησιµοποιήθηκαν είναι όλα open source. Για όλους τους
		χρήστες που θα χρησιµοποιήσουν αυτή την εφαρµογή το µόνο που απαιτείται είναι
		ένα smartphone µε οποιοδήποτε λειτουργικό, και σύνδεση στο ∆ιαδίκτυο για την 
		εγγραφή τους στην εφαρμογή. Στην συνέχεια δεν είναι απαραίτητη η πρόσβαση στο
		Διαδύκτιο για την χρήση της εφαρμογής. 
		Για την ανάλυση απαιτήσεων έγινε αρχικά µια
		έρευνα, όπου µελετήθηκαν παρόµοια συστήµατα.

		\textbf{∆οµή εφαρµογής:}

 		Για τη διαγραµµατική απεικόνιση της δοµής της εφαρµογής κατά τη φάση της
		σχεδίασης χρησιµοποιήθηκε το εργαλείο λογισµικού \textbf{draw.io} με το οποίο
		σχεδιάστηκε το αρχικό διάγραµµα ροής προγράμματος (flowchart) για τη δοµή της εφαρµογής. 
		
		Για την ευκολότερη κατανόηση της δοµής της εφαρµογής ακολουθεί η παρακάτω εικόνα:

		\newpage
		\begin{figure}[!htb]
			\caption{Διάγραμμα ροής προγράμματος.}
			\vspace*{0.5cm}
			\centering
			\includegraphics[width=0.9\linewidth]{a1}
		\end{figure}

		\subsection{Υλοποίηση}
			\subsubsection{Εγκατάσταση Λογισμικού}

				\myparagraph{Eγκατασταση Node}
				Node είναι ένα περιβάλλον χρόνου εκτέλεσης που επιτρέπει την εγγραφή της JavaScript
				στην πλευρά του διακομιστή. Εκτός από το ότι χρησιμοποιείται για υπηρεσίες ιστού, 
				το Node χρησιμοποιείται συχνά για την ανάπτυξη εργαλείων προγραμματιστών, 
				όπως το Ionic CLI.

				\begin{figure}[!htb]
					\begin{center}
						\caption{Εγκατάσταση Node.js Νο.1}
						\vspace*{0.5cm}
						\includegraphics[width=0.9\linewidth]{nod0} 
					\end{center}
				\end{figure}

				\begin{figure}[!htb]
					\begin{center}
						\caption{Εγκατάσταση Node.js Νο.2}
						\vspace*{0.5cm}
						\includegraphics[width=0.9\linewidth]{nod} 
					\end{center}
				\end{figure}

				\begin{figure}[!htb]
					\begin{center}
						\caption{Εγκατάσταση Node.js Νο.3}
						\vspace*{0.5cm}
						\includegraphics[width=0.9\linewidth]{nod2} 
					\end{center}
				\end{figure}
				\clearpage
				\myparagraph{Εγκατάσταση Visual Studio Code}
				Είναι στο χέρι του καθενός να διαλέξει ποιόν επεξεργαστή κειμένου θα χρησιµοποιήσει. Εγώ διάλεξα το Visual Studio Code το οποίο
				παρέχει υπέρ αρκετές δυνατότητες στο χρήστη. Παρακάτω φαίνεται πως μπορεί να γίνει η εγκατάσταση του Visual Studio Code.
				
				\begin{figure}[!htb]
					\begin{center}
						\caption{Εγκατάσταση Visual Studio Code Νο.1}
						\vspace*{0.5cm}
						\includegraphics[width=0.9\linewidth]{vsc1} 
					\end{center}
				\end{figure}

				\begin{figure}[!htb]
					\begin{center}
						\caption{Εγκατάσταση Visual Studio Code Νο.2}
						\vspace*{0.5cm}
						\includegraphics[width=0.9\linewidth]{vsc} 
					\end{center}
				\end{figure}

				\newpage
				Αφου εγκατασταθεί το Visual Studio Code επόμενο βήμα είναι η εγκατάσταση του npm.
				
				\myparagraph{Εγκατάσταση npm}
				Npm είναι ο διαχειριστής πακέτων του Node. Επιτρέπει στους προγραμματιστές να εγκαταστήσουν, να μοιραστούν και να συσκευάσουν Node modules (δομοστοιχεία).  
				Το Ionic, μαζί με τις εξαρτήσεις του μπορεί να εγκατασταθεί με το npm.
				
				\begin{figure}[!htb]
					\begin{center}
						\caption{Εγκατάσταση npm Νο.1}
						\vspace*{0.5cm}
						\includegraphics[width=\linewidth]{npm1} 
					\end{center}
				\end{figure}

				\begin{figure}[!htb]
					\begin{center}
						\caption{Εγκατάσταση npm Νο.2}
						\vspace*{0.5cm}
						\includegraphics[width=\linewidth]{npm2} 
					\end{center}
				\end{figure}
				Επόμενο βήμα η εγκατάσταση του Ionic.
	
				\myparagraph{Εγκατάσταση Ionic 4} 
				Η εγκατάσταση του Ionic γίνεται με αυτήν την εντολή \textbf{npm install -g ionic}.
				Αφού γίνει η εγκατάσταση χωρίς να εμφανιστεί κάποιο σφάλμα, με την εντολή \textbf{ionic start myApp} δημιουργείται μια νέα εφαρμογή με το 
				όνομα "\textbf{myApp}". Στη συνέχεια το Ionic  δίνει τη δυνατότητα να επιλογής template (προτύπου). Αφου γίνει ή όχι η επιλογή κάποιου template, 
				ξεκινάει να γίνεται η λήψη των απαραίτητων στοιχείων του Ionic.
				
				\begin{figure}[!htb]
					\begin{center}
						\caption{Εγκατάσταση Ionic Νο.1}
						\vspace*{0.5cm}

						\includegraphics[width=\linewidth]{ionic1} 
					\end{center}
				\end{figure}

				\begin{figure}[!htb]
				\begin{center}
					\caption{Εγκατάσταση Ionic Νο.2}
					\vspace*{0.5cm}
						\includegraphics[width=\linewidth]{ionic2} 
					\end{center}
				\end{figure}

				\begin{figure}[!htb]
					\begin{center}
						\caption{Εγκατάσταση Ionic Νο.3}
						\vspace*{0.5cm}
						\includegraphics[width=\linewidth]{ionic3} 
					\end{center}
				\end{figure}
				\clearpage

				Με την εντολή \textbf{ionic start myApp} το Ionic εγκαθιστά απαραίτητες εξαρτήσεις και δημιουργεί κάποιους φακέλους.
				Παρακάτω απεικονίζεται η δομή έργου ενός νέου Ionic project:
		
				\setlength\intextsep{0pt}
				\vspace*{1cm}
				\begin{wrapfigure}{l}{0.45\textwidth}
					\caption{Δομή έργου Ionic}
					\vspace*{0.5cm}
					\includegraphics[width=0.45\textwidth]{ionic4}
					
				\end{wrapfigure}
					\textbf{e2e} είναι τα αρχικά του End to End testing είναι ένας τρόπος για να βεβαιωθούμε οτι η εφαρμογή 
					 μας λειτουργεί σωστά. Συνήθως χρησιμοποιούμε τις δοκιμές E2E για να διασφαλίσουμε οτι τα στοιχεία μας
					 λειτουργούν σωστά μαζί.

					\textbf{node modules} παρέχει πακέτα npm σε ολόκληρο το χώρο εργασίας.
					
					\textbf{src} περιέχει αρχεία προέλευσης για το έργο εφαρμογής σε επίπεδο root.
					
					\textbf{angular.json} περιέχει προεπιλεγμένες ρυθμίσεις παραμέτρων CLI για όλα 
					τα έργα στο χώρο 
					εργασίας συμπεριλαμβανομένων των επιλογών διαμόρφωσης για τα εργαλεία δημιουργίας, προβολής και δοκιμής που χρισημοποιεί
					το CLI όπως το \textbf{tslint}.

					\textbf{ionic.cofig.json} περιέχει αρχεία διαμόρφωσης του έργου.	

					\textbf{package-lock.json} παρέχει πληροφορίες έκδοσης για όλα τα πακέτα που έχο-
					υν εγκατασταθεί στο φάκελο node modules 
					από το npm. 

					\textbf{package.json} ρυθμίζει τις εξαρτήσεις πακέτων npm που είναι διαθέσιμες σε όλα τα έργα στο χώρο εργασίας. 

					\textbf{tsconfig.json} είναι μία προεπιλεγμένη διαμόρφωση TypeScript για έργα στο χώρο εργασίας.

					\textbf{tslint.json} είναι ένα εργαλείο στατικής ανάλυσης που ελέγχει τον κώδικα Typ-
					eScript για σφάλματα ευκρίνειας, συντήρησης και λειτουργικότητας.
					\newpage

					Με την εντολή \textbf{ionic serve} γίνεται η εκίνηση ενός τοπικού server για τον έλεγχο της εφαρμογής. Ο server τρέχει στον browser και παρακολουθεί 
					τις αλλαγές στα source files και επαναφορτώνει αυτόματα με την ενημερωμένη έκδοση. Επίσης δίνει τη δυνατότητα στο χρήστη να έχει μια εικόνα για το
					πως φαίνεται η εφαρμογή σε κινητές συσκευές με διαφορετικά λειτουργικά συστήματα. Αυτό είναι πολύ χρήσιμο γιατί πολλές φορές κάτι που λειτουργεί και φαίνεται 
					σωστά στο ένα λειτουργικό στο άλλο παρουσίαζει σφάλματα.

					Παρακάτω μια εικόνα με το τι εμφανίζεται στον browser μετά την εντολή:
					\vspace*{1cm}
					
				\begin{figure}[!htb]
					\begin{center}
						\caption{ionic serve command}
						\vspace*{0.5cm}
						\includegraphics[width=\linewidth]{ionicServe} 
					\end{center}
				\end{figure}

		\newpage
		\subsection{Έλεγχος}

		Η φάση του ελέγχου ακολουθεί µετά από τη φάση της υλοποίησης για να εντοπιστούν
		εντοπιστούν παραλέιψεις ή δυσλειτουργίες. Αν υπάρξει κάτι από τα προηγούμενα  γινεταό επιστροφή στη φάση της υλοποίησης
		ή ακόµα και σε κάποια προγενέστερη φάση αν κρίνεται απαραίτητο. Σκοπός της φάσης αυτής είναι
		πάντοτε η βέλτιστη τελική κατάσταση που μπορεί να φτάσει το σύστημα.

		Στη συγκεκριµένη εφαρµογή ο έλεγχος έγινε χρησιµοποιώντας την
		εφαρµογή σε Android και iOS συσκευές. Σε όλες τις
		περιπτώσεις η εφαρµογή λειτούργησε ορθά και δεν παρουσιάστηκαν σφάλµατα.
		
		\subsection{∆ιανοµή}

		Η παρόυσα εφαρµογή δε θα διατίθεται κάπου προς πώληση ή εγκατάσταση (App Store, Google Play Store, Microsoft Store).
		Σχεδιάστηκε και υλοποιήθηκε µόνο στα πλάισια της πτυχιακής εργασίας. 
		\newpage
		\section{Παρουσίαση Εφαρμογής με περιγραφή κώδικα}
			Στο κεφάλαιο αυτό γίνεται η περιγραφή του κώδικα καποίων βασικών λειτουργιών της εφαρμογής ώστε να γίνει κατανοητή η υλοποίηση της στον αναγώνστη.
			\subsection{Δημιουργία συστήματος ελέγχου ταυτότητας χρήστη}

				Το σημαντικότερο χαρακτηριστικό για τις περισσότερες εφαρμογές είναι το σύστημα ελέγχου ταυτότητας χρήστη. Για τη λειτουργία αυτή χρησιμοποιήθηκε 
				η Firebase της Google, μια υπηρεσία που παρέχει δυνατότητες βάσης δεδομένων, ελέγχου ταυτότητας χρήστη και πολλών άλλων. 
				Τα βήματα μέσα στή Firebase είναι πρώτον η δημιουργία ενός νέου project και στη συνέχεια η επιλογή της μεθόδου σύνδεσης.
				
				Παρακάτω υπάρχουυν εικόνες που δείχνουν ακριβώς αυτό:

				\vspace*{1cm}

			\begin{figure}[!htb]
				\begin{center}
					\caption{Firebase Singu-in method}
					\vspace*{0.5cm}

					\includegraphics[width=0.9\linewidth]{auth} 
				\end{center}
			\end{figure}

			\begin{figure}[!htb]
				\begin{center}
					\caption{Firebase Authentication}
					\vspace*{0.5cm}

					\includegraphics[width=0.9\linewidth]{firebaseNo1} 
				\end{center}
			\end{figure}

			\newpage
			Για να λειτουργήσει η Firebase με την εγαρμογή δημιουργήθηκε μια Global\newline
			Angular Service. Σε αυτή αρχικά γίνεται εισαγωγή διάφορων στοιχείων.  
			Του Ang-
			ular Router για να γίνει η ανακατέυθυνση
			του χρήστη όταν αποσυνδεθεί. Στη συνέχεια του AngularFireAuth για την αλληλεπίδραση με τη Firebase Auth και του Fire Store. Η υπηρεσία αυτή έχει ένα κομμάτι
			πληροφοριών που μπορεί να μοιράζεται σε όλλα τα εξαρτήματα, αυτό το κομμάτι είναι η καταγραφή του εγγράφου του χρήστη στη βάση δεδομένων. Ορίζεται ως  \textbf{Observable} γιατί
			μπορεί να αλλάξει κάθε φορά που ο χρήστης συνδέεται και αποσυνδέεται. Στη λειτουργία \textbf{singupUser} γίνεται η επιλογή του τρόπου εγγραφής του χρήστη, στη συγκεκριμένη 
			περίπτωση με Email και Password. Στη συνέχεια δημιουργείται στη βάση δεδομένων νέο πιστοποιητηκό χρήστη με user ID και user Email. Με τη λειτουργία 
			\textbf{ressetPassword} ο χρήστης εισάγει το email του και η Firebase του στέλνει αυτόματα ένα email επαναφοράς κωδικού. Τέλος η λέιτουργία \textbf{logoutUser} αποσυνδέει το χρήστη από
			τη firebase και τον καθοδιγεί στη σελίδα \textbf{login}.

			Παρακάτω υπάρχει ο κώδικας της υπηρεσίας αυτής:

			\begin{figure}[!htb]
				\begin{center}
					\caption{Authendication Service}
					\vspace*{0.5cm}

					\includegraphics[width=0.9\linewidth]{authService} 
				\end{center}
			\end{figure}

			\newpage
			Στη συνέχεια υπάρχει ακόμα μια υπηρεσία, η Authentication Guard Service η οποία ευθύνεται για την ασφάλεια της εφαρμογής. Ελέγχει αν ο χρήστης είναι αποσυνδεδεμένος η όχι όταν ανόιγει την εφαρμογή και πράτει ανάλογα.

			Η μέθοδος \textbf{canActivate} επιστρέφει μια boolean ή μια Observable boolean τιμή, αν είναι true ενεργοποιεί τη 
			διαδρομή (route) και αν είναι false όχι. Για αυτό έγινε η μετατροπή του user Observable, που ορίστηκε στην authentication service, σε boolean format.
			Μέσα στη μέθοδο υπάρχει πρώτα ο operator \textbf{take(1)} ο οποίος ολοκληρώνει το Observable αφού μεταδωθεί η πρώτη τιμή, γιατί δεν θέλουμε να συνεχίσει να δουλεύει 
			αφού η διαδρομή μπλόκαριστεί. Αμέσως μετά αντιστοιχίζεται το αντικείμενο σε boolean με τον \textbf{map} operator. Τέλος αν ο χρήστης δεν είναι συνδεδεμένος 
			καθοδηγείται στη σελίδα σύνδεσης και του εμφανίζεται μια ειδόποίηση με τη \textbf{presentAlert} λειτουργία, λέγοντας του οτι πρέπει να συνδεθεί για να συνεχίσει.

			\newpage
			\begin{figure}[!htb]
				\begin{center}
					\caption{Authentication Guard Service}
					\vspace*{0.5cm}

					\includegraphics[width=0.9\linewidth]{authGuard} 
				\end{center}
			\end{figure}

			\newpage
			\subsection{Home Page}

			\begin{figure}[!htb]
				\caption{Home Page}
				\vspace*{0.5cm}

				\minipage{0.32\textwidth}
				  \includegraphics[width=\linewidth]{home1}
				  
				\endminipage\hfill
				\minipage{0.32\textwidth}
				  \includegraphics[width=\linewidth]{home2}
				
				\endminipage\hfill
				\minipage{0.32\textwidth}%
				  \includegraphics[width=\linewidth]{home3}
				  
				\endminipage
			\end{figure}

			\vspace*{1cm}


			Παραπάνω απεικονίζεται η αρχική σελίδα. Αποτελείται από ένα \textbf{slider} το οπο-
			ίο δίνει τη δυνατότητα στο χρήστη να διαλέξει 
			το επίπεδο δυσκολίας των ασκήσεων και να φτιάξει το δικό του πλάνο γυμναστικής. Επίσης υπάρχουν τα κουμπία \textbf{lesson} τα οποία μεταφέρουν το χρήστη στη σελίδα των ασκήσεων. Στην 
			προκειμένη περίπτωση το lesson 1 του επιπέδου begginer είναι checked και ο χρήστης δεν μπορεί να το επιλέξει, αυτό γιατί έχει 
			ολοκληρώσει το μάθημα αυτό. 

			Στη συνέχεια υπάρχουν οι βασικές λειτουργίες την αρχικής σελίδας, από τη μεριά του κώδικα.
			\vspace*{1cm}

			\begin{wrapfigure}{l}{0.45\textwidth}
				\caption{Home page imports}
				\vspace*{0.5cm}

				\includegraphics[width=0.45\textwidth]{homeImports}	
			\end{wrapfigure}
			
			Στην αρχή του κώδικα βλέπουμε τα απαραίτητα στοιχεία που πρέπει να εισάγουμε για τις διάφορες λειτουργίες της αρχικής σελίδας. \textbf{Router} για την πλοήγηση του χρήστη απο σελίδα σε σελίδα. 
			\textbf{NavigationExtras} για τη μεταφορά δεδομένων από σελίδα σε σελίδα.
			\textbf{Network} για πληροφορίες δικτύου. \textbf{Auth2Service} για πληροφορίες χρήστη. \textbf{Storage} για τη χρήση του αποθηκευτικού 
			χώρου της συσκευής. \textbf{AlertController, ToastController} για την εμφάνιση ειδοποιήσεων και τέλος \textbf{DataService} μια Global 
			Service που δημιουργήθηκε για τη διαχείρηση δεδομένων όλης της εφαρμογής.
			\vspace*{1cm}

			\begin{wrapfigure}{l}{0.45\textwidth}
				\caption{Constructor και ngOnInit}
				\vspace*{0.5cm}

				\includegraphics[width=0.45\textwidth]{homInit}	
			\end{wrapfigure}

			Ο \textbf{contstructor} είναι η προεπιλεγμένη μέθοδος της κλάσης που εκτελείται όταν η κλάση δημιουργείται και εξασφαλίζει τη σωστή προετοιμασία πεδίων στην κλάση και τις υποκατηγορίες της.
			Συνήθως η \textbf{ngOnInit} χρησιμοποιείται για όλη την αρχικοποίηση / δήλωση. Μέσα στη ngOnInit η πρώτη γραμμή κώδικα καλείται κάθε φορά που έχει γίνει μια αλλαγή στον αποθηκευτικό χώρο της συσκευής, για παράδειγμα όταν ο χρήστης ολοκληρώνει ένα lesson, η αρχική σελίδα πρέπει να γνωρίζει πότε έγινε αυτή η αλλαγή για να απενεργοποιήσει το ανάλογο κουμπί όπως είδαμε παραπάνω. Καλείται η
			μέθοδος \textbf{watchStorage} της \textbf{DataService}. Οι συναρτήσεις \textbf{checkForUser} και \textbf{check-
			ForPlan} καλούνται κάθε φορά που γίνεται η εμφάνιση της σελίδας, και όταν γίνει μια αλλαγή στον αποθηκευτικό χώρο. Στη συνέχεια γίνεται ενημέρωση ανάλογα με την κατάσταση του δικτύου.

			\newpage
			\begin{wrapfigure}{l}{0.45\textwidth}
				
				\caption{Συνάρτηση checkForUser}
				\vspace*{0.5cm}

				\includegraphics[width=0.45\textwidth]{checkUser}	
			\end{wrapfigure}

			Στη συνάρτηση αυτή πρώτα απο όλα, γίνεται ο έλεγχος χρήστη, αφού έχουμε πάρει τα στοιχεία του χρήστη και πιο συγκεκριμένα το \textbf{user.uid}, δημιουργούμε τρείς σταθερές (UidBegginer, UidIntermadiate και UidAdvanced) για να ελέγξουμε τον αποθηκευτικό χώρο. Με την εντολή \newline
			\textbf{storage.get(UidBegginer)} παίρνουμε τα δεδομένα για το συγκεκριμένο \textbf{ID}. Γεμίζουμε τον πίνακα \textbf{BegginerLessons} με τιμές τέτοιες ώστε ο χρήστης να μπορέι να πατήσει
			 όλα τα κουμπία \textbf{lesson}, για αυτό και αρχικά η τιμή είναι \textbf{disabled: false}. Στη συνέχεια αν το μάθημα έχει ολοκληρωθεί ψάχνουμε μέσα στον πίνακα BegginerLessons και αλλάζουμε την τιμή disabled του συγκεκριμένου κουμπιού σε true, για να μην μπορεί να το επιλέξει ο χρήστης. Επειτα περνάμε στην HTML τον πίνακα.

			\vspace*{1cm}

			\begin{figure}[!htb]
				\begin{center}
					\caption{Home page HTML example}
					\vspace*{0.5cm}

					\includegraphics[width=\linewidth]{homeHtml} 
				\end{center}
			\end{figure}

			Παρακάτω βλπέπουμε το τέταρτο μέρος του slider το οποίο δίνει τη δυνατότητα στο χρήστη να φτίαξει το δικό του πλάνο εξάσκησης:
			
			\newpage
			Οταν ο χρήστης πατήσει το κουμπί μεταφέρεται στη σελίδα \textbf{Add Exercise} στην οποία υπάρχουν όλες οι ασκήσεις και μπορεί να τις προσθέσει 
			στο πλάνο του πατώντας το κουμπί στα δεξία της άσκησης. Στη συνέχεια μεταφέρεται σε μία νέα σελίδα που υπάρχει η συγκεκριμένη άσκηση με την εξήγηση της και αν πατήσει το κουμπί
			στο κάτω μέρος της σελίδας. Στο παρασκήνιο δημιουργείται ένας πίνακας με τη συγκεκριµένη άσκηση στη μνήμη της συσκευής με το κάλεσμα της \textbf{DataService},
			ο χρήστης μεταφέρεται στη σελίδα \textbf{New Training}
			η οποία ελέγχει τη μνήμη της συσκευής για αλλαγές και αν εντοπίσει κάποια την εμφανίζει αυτόματα στο χρήστη. Επίσης
			δίνει στο χρήστη τη δυνατότητα να αφαιρέσει και να προσθέσει ασκήσεις. Αν επιλέξει να προσθέσει κάποια, πατώντας το κουμπί
			μεταφέρεται στη σελίδα \textbf{Add Exercise} και επαναλαμβάνει τη διαδικασία. Αν επιλέξει να αφαιρέσει, καλείται πάλι 
			η \textbf{DataService} και διαγράφει την άσκηση απο τον πίνακα.
			
			\vspace*{1cm}

			\begin{figure}[!htb]
				\caption{Home Page user's custom training No.1}
				\vspace*{0.5cm}

				\minipage{0.24\textwidth}
				  \includegraphics[width=\linewidth]{plan1}
				  
				\endminipage\hfill
				\minipage{0.24\textwidth}
				  \includegraphics[width=\linewidth]{plan2}
				
				\endminipage\hfill
				\minipage{0.24\textwidth}%
				  \includegraphics[width=\linewidth]{plan3}
				  
				\endminipage\hfill
				\minipage{0.24\textwidth}%
				  \includegraphics[width=\linewidth]{plan4}
				  
				\endminipage
			\end{figure}

			\newpage

			Αποθηκεύει το πλάνο του πατώντας το κουμπί Save και του ζητείται ένα όνομα για το πλάνο. Με το που δώσει 
			κάποιο όνομα μεταφέρεται στη Home Page όπου μπορεί να επιλέξει το πλάνο του ανα πάσα στιγμή.
			\vspace*{1cm}

			\begin{figure}[!htb]
				\caption{Home Page user's custom training Νο.1}
				\vspace*{0.5cm}

				\minipage{0.48\textwidth}
				  \includegraphics[width=\linewidth]{plan5}
				  
				\endminipage\hfill
				\minipage{0.48\textwidth}
				  \includegraphics[width=\linewidth]{plan6}
				\endminipage\hfill

			\end{figure}
			\vspace*{1cm}
			Είναι σημαντικό να αναφερθεί οτι για να γίνουν αυτές οι αλλαγές αυτόματα, έχουν δημιουργηθεί μια σειρά από 
			\textbf{Observables} στη \textbf{DataService}, τα οποία πυροδοτούνται με κάθε αλλαγή του \textbf{Local-Storage} και ενημερώνουν όλα τα στοιχεία της εφαρμογής. 
			Χωρίς αυτά για να έβλεπε της αλλαγές ο χρήστης θα έπρεπε κάθε φορά να ανανεώνει τη σελίδα, κάτι που θα έκανε 
			την εφαρμογή αδύνατη για χρήση.
		
			Περνώντας στο κομμάτι των μαθημάτων παρακάτω βλέπουμε τη διαδικασία που ακολουθεί ο χρήστης.
			\newpage
			\begin{figure}[!htb]
				\caption{Lesson Page Νο.1}
				\vspace*{0.5cm}

				\minipage{0.49\textwidth}
				  \includegraphics[width=\linewidth]{home2}
				  
				\endminipage\hfill
				\minipage{0.49\textwidth}
				  \includegraphics[width=\linewidth]{lesson1}
				\endminipage\hfill
			
			\end{figure}
			\vspace*{1cm}

			\begin{wrapfigure}{l}{0.45\textwidth}
				\caption{Κώδικας Home Page}
				\vspace*{0.5cm}
				\includegraphics[width=0.45\textwidth]{navig}
				
				\caption{Κώδικας Lesson Page}
				\vspace*{0.5cm}
				\includegraphics[width=0.45\textwidth]{navig2}
				
			\end{wrapfigure}
			Οταν ο χρήστης επιλέγει ένα μάθημα από κάποιο πρόγραμμα, μεταφέρεται στην ανλαόγη σελίδα (begginer, intermadiate ή advnanced), το περιεχόμενο της οποίας αλλάζει δυναμικά,  για παράδειγμα ο αριθμός του μαθήματος 
			στο πάνω μέρος της σελίδας παρέχεται από τη home page μέσω της διαδρομής που επιλέγει ο χρήστης. Και στη σελίδα που μεταφέρεται λαμβάνει το id
			και το χρησιμοποιεί αναλόγως. Για καλύτερη κατανόηση υπάρχουν τα σχήματα 29 και 30.
		
			\newpage	
			Ο χρήστης ξεκινάει το πρόγραμμα πατώντας το κουμπί στο κάτω μέρος της σελίδας και μεταφέρεται σε μια νέα σελίδα η οποία αποτελέιται από ένα slider
			με της ασκήσεις. Για να συνεχίσει έχει δύο επιλογές το κουμπί Skip και Done, αναλόγως το κουμπί προστίθενται ή όχι θερμίδες μέχρι και την τελευταία
			άσκηση. Στην συνέχεια μεταφέρεται σε μια σελίδα όπου του ζητείται το βάρος του, αν θέλει το βάζει και μεταφέρεται στην Profile Page στην οποία μπορέι να 
			δει τα διαγράμματα των θερμίδων και του βάρους του.
			\vspace{.5cm}

			\begin{figure}[!htb]
				\caption{Lesson page}
				\vspace*{0.5cm}

				\minipage{0.24\textwidth}
				  \includegraphics[width=\linewidth]{lesson2}
				  
				\endminipage\hfill
				\minipage{0.24\textwidth}
				  \includegraphics[width=\linewidth]{lesson3}
				\endminipage\hfill
				\minipage{0.24\textwidth}
				  \includegraphics[width=\linewidth]{lesson4}
				\endminipage\hfill
				\minipage{0.24\textwidth}
				  \includegraphics[width=\linewidth]{lesson5}
				\endminipage\hfill
			
			\end{figure}
			\vspace{.5cm}
			Στην σελίδα τών ασκήσεων βλέπουμε το κομμάτι του κώδικα που ευθύνεται για την διαχείρηση των θερμίδων. Πατώντας το κουμπί Done
			προστίθενται σε ένα άθροισμα οι θερμίδες κάθε άσκησης, που βρίσκονται σε ένα .json αρχείο, με το κουμπί Skip δίνεται στη συνάρτηση \textbf{Done}
			μηδέν ως τιμή της συνάρτησης, για να μην επιρεάζει το άθροισμα.
			\vspace{1cm}

			\begin{figure}[!htb]
					\begin{center}
						\caption{Lesson page HTML example}
						\vspace*{0.5cm}
						\includegraphics[width=.9\linewidth]{doneHtml} 
					\end{center}  	
				\end{figure}

			\newpage
			Στη μεριά της TypeScript η συνάρτηση \textbf{Done} καλείται κάθε φορά που πατιέται ένα από τα κουμπία.
			Οι θερμίδες προστίθενται σε μια μεταβλητη όσο ο αριθμός τον αντικειμένων είναι μικρότερος από το μήκος του
			πίνακα των ασκήσεων. Οταν ολοκληρωθεί το πρόγραμμα δημιουργείται ο πίνακας \textbf{ChartData}
			με τις τρείς τιμές που βλέπουμε και δίνεται ως τιμή στην συνάρτηση \textbf{setData} της \textbf{DataService} για να αποθηκευτεί
			στη μνήμη της συσκευής. Τέλος η \textbf{DataService} πυροδοτεί το ανάλογο \textbf{Observale} και ενημερώνεται η Home Page, για την απενεργοποίηση του κουμιού, και
			η Profile Page για την ένδειξη του διαγράμματος των θερμίδων.
			\vspace{.5cm}

			\begin{figure}[!htb]
				\begin{center}
					\caption{Lesson page Typescript example}
					\vspace*{0.5cm}
					\includegraphics[width=.9\linewidth]{doneTS} 
				\end{center}
			\end{figure}

		\newpage
		Οταν ο χρήστης προηγηθεί στην Profile Page βλέπει τα διαγράμματα θερμίδων και βάρους. Στο σχήμα που ακολουθεί βλέπουμε τα διαγράμματα που έχουν δημιουργηθεί
		για διαφορετικα ενδεχόμενα τριών ολοκληρωµένων μαθημάτων.
		\vspace*{1cm}
		\begin{figure}[!htb]
			\caption{Profile page example}
			\vspace*{0.5cm}

			\minipage{0.33\textwidth}
			  \includegraphics[width=\linewidth]{lesson5}
			\endminipage\hfill
			\minipage{0.33\textwidth}
			  \includegraphics[width=\linewidth]{lesson6}
			\endminipage\hfill
			\minipage{0.33\textwidth}
			  \includegraphics[width=\linewidth]{profile4}
			\endminipage\hfill
		
		\end{figure}

		\newpage
		\section{ Συµπεράσµατα και Μελλοντικές Επεκτάσεις }	
					
		\section{Δικτυογραφία-Βιβλιογραφία}
			Panhale, Mahesh. “Beginning Hybrid Mobile Application Development,” n.d., 229.

			Ravulavaru, Arvind. Learning Ionic: Build Hybrid Mobile Applications with HTML5, SCSS, and Angular, 2017.

			Arvind Ravulavaru Learning Ionic Build Real-Time and Hybrid 
			Mobile Appl
			ications with Ionic-Packt Publishing 2015
			
			Coury, Felipe, Ari Lerner, Nate Murray, and Carlos Taborda. Ng-Book: The Complete Guide to Angular 4, 2017.

			Huber, Thomas Claudius. Getting Started with TypeScript: Includes ,
			Intro
			duction to Angular ; Write Professional JavaScript Code That Scales, Use Interfac
			es and Classes to Build Robust Code, 
			Learn about Generics, Modules, Arrow Functi
			ons, Decorators, Declarations, Npm and Much More. 1st edition. 
			Erscheinu
			ngsort nicht ermittelbar: Thomas Claudius Huber, 2017.

			https://cordova.apache.org/docs/en/latest/guide/overview/

			https://www.draw.io/

			https://www.telerik.com/blogs/what-is-a-hybrid-mobile-app-
			
			https://nordicapis.com/what-is-the-difference-between-an-api-and-an-sdk/ 
			
			https://blog.codecentric.de/en/2014/11/ionic-angularjs-framework-on-the-ris
			
			e/
			
			https://www.freecodecamp.org/news/a-deeply-detailed-but-never
			-definitive-

			guide-to-mobile-development-architecture-6b01ce3b1528/
			
			https://www.webmd.com/diet/ss/slideshow-best-diet-tips-ever
			
			https://medium.com/@javier.ramos1/ionic-4-all-you-need-to-know-d2b9627
			
			aaf03

			https://www.quora.com/What-is-AngularJS-in-simple-words
			
			
\end{document}